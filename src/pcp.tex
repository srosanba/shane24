\documentclass[9pt]{article}
\usepackage{statrep}
\def\SRrootdir{H:/GraphicsGroup/StatRep/explore/pcp}
\usepackage[left=1.0in,right=1.0in,top=0.5in,bottom=1.0in]{geometry}
\usepackage[none]{hyphenat}
\renewcommand{\rmdefault}{phv} % Arial
\renewcommand{\sfdefault}{phv} % Arial
\setlength{\parindent}{0pt}
  
\title{Parallel Coordinates Plots Made Easy}
\author{Shane Rosanbalm, Rho, Inc.}
\date{\vspace{-5ex}}

\begin{document}

\maketitle

\section*{Abstract}

A parallel coordinates plot is useful for visualizing multivariate data. Unfortunately, there isn't a PARALLEL statement in SGPLOT. In this paper we present a macro called \texttt{\%parallel}. Using a minimum of parameters \texttt{(data=, var=, group=)} the macro will produce a parallel coordinates plot via SGPLOT. 

\section*{Background}

Producing a parallel coordinates plot in SAS is not straightforward. There definitely isn't a PARALLEL statement in SGPLOT. The best approach I could find online was from SAS author Prashant Hebbar in his paper from SGF 2012 (Off the Beaten Path: Creating Unusual Graphs with GTL). The outlined process certainly works, but it's written for readability and not for flexibility or scalability. I decided to experiment to see if was possible to generate a parallel coordinates plot using more flexible and scalable code. The result of this experiment is a macro called \texttt{\%parallel} which is capable of producing a parallel coordinates plot with a minimum of parameters.

\section*{Macro Basics}

A basic call to \texttt{\%parallel} looks like this: \\

\begin{Sascode}[program]
options sasautos=("H:/GitHub/srosanba/sas-parallel-coordinates-plot/src" sasautos);
ods graphics / height=3in;
\end{Sascode}

\begin{Sascode}[store=basics]
%parallel
   (data=sashelp.iris
   ,var=sepallength sepalwidth petallength petalwidth
   ,group=species
   );
\end{Sascode}

\Graphic[store=basics,
   objects=SGPlot,
   caption={Parallel Coordinates Plot for Fisher's Iris Data}
   ]{basics}

The required parameters are \texttt{data=} and \texttt{var=}.

\begin{tabular}{|l|l|}

\hline
Parameter & Description \\ \hline
\texttt{data=} & \begin{tabular}{@{}l@{}} Input dataset. \\ Required. \end{tabular} \\ \hline
\texttt{var=} & \begin{tabular}{@{}l@{}} Space-separated list of variables to plot. \\ Required. \end{tabular} \\ \hline

\end{tabular} \\ \\

The optional parameters that are likely to be of most interest are \texttt{group=} and \texttt{axistype=}.

\begin{tabular}{|l|l|}

\hline
Parameter & Description \\ \hline
\texttt{group=} & \begin{tabular}{@{}l@{}} Grouping variable. \\ Optional. \end{tabular} \\ \hline
\texttt{axistype=} & \begin{tabular}{@{}l@{}} Type of yaxis to create. \\ Optional. \\ Valid values: \texttt{percentiles | datavalues}. \end{tabular} \\ \hline

\end{tabular} \\ \\

Using \texttt{axistype=datavalues} changes the yaxis of the previous output as follows:

\begin{Sascode}[program]
ods graphics / height=3in;
\end{Sascode}

\begin{Sascode}[store=datavalues]
%parallel
   (data=sashelp.iris
   ,var=sepallength sepalwidth petallength petalwidth
   ,group=species
   ,axistype=datavalues
   );
\end{Sascode}

\Graphic[store=datavalues,
   objects=SGPlot,
   caption={Parallel Coordinates Plot Using Data Values}
   ]{datavalues}




\end{document}